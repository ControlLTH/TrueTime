% Dokumentdatablad f�r rapporter
% ==============================
% Uppgifterna fylls i p� engelska inom {}.
% De prickade raderna skall alltid fyllas i (och �verfl�diga prickar tas bort),
% �vriga uppgifter �r beroende av rapportens art.
%
% \name          Ex: FINAL REPORT , MASTER THESIS , INTERNAL REPORT
% \date          M�nad och �rtal, ex: October 1985
% \num           Nummer (av Britt-Marie), dvs endast siffrorna.
% \author        Namn p� f�rfattare
% \supervisor    Handledare
% \title         Rapportens titel (ev med engelsk �vers.)
% \keywords      Eventuella nyckelord
% \language      Det spr�k rapporten �r skriven p�
% \pages         Totalt antal sidor
% \begin{abstract}...\end{abstract} Abstract p� engelska
%
%------------------------------------------------------------------------------
%
\documentclass{docdata}
\usepackage{schoolbook}
\begin{document}
\name{INTERNAL REPORT}
\date{October 2003}
\num{7605}
\author{Dan Henriksson, Anton Cervin}
\supervisor{}
\sponsor{}
\title{\textsc{TrueTime} 1.13---Reference Manual}
\keywords{Real-time control systems, Event-based simulation, Shared resources,
Real-time kernel, Feedback sched\-uling, Networked control systems.}
\classification{}
\supplement{}
\isbn{}
\language{English}
\pages{73}
\security{}
\recipient{}
\begin{abstract}
  The manual describes the use of TrueTime, a Matlab/Simulink-based
  tool for simulation of distributed real-time control systems. The
  tool facilitates detailed co-simulation of plant dynamics,
  controller task execution, and network transmissions. The TrueTime
  Kernel and TrueTime Network blocks are described, and the real-time
  kernel primitives are detailed.
\end{abstract}
\end{document}
